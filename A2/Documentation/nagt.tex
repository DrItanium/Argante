\documentclass[a4paper,oneside,openany]{book}
\title{\Huge{Argante2}\\\huge{Assembly Tutorial}}
\author{
	\LARGE{James Kehl}\\
	ecks@optusnet.com.au
}
\begin{document}
\maketitle{}
\newcommand{\email}[1]{$<$#1$>$}
\newcommand{\FIXME}{\texttt{**FIXME**}}
\newcommand{\perjsk}{James Kehl \email{ecks@optusnet.com.au}}
\newcommand{\permz}{Michal Zalewski \email{lcamtuf@coredump.cx}}
\newcommand{\bsz}{\^ }
\newcommand{\rarrow}{$\rightarrow$}

\newenvironment{codefile}[1]{
	\smallskip\\\ttfamily begin #1.agt \hrulefill
	\begin{tabbing}
	\enskip\=\qquad\qquad\=\+\kill
}
{
	\end{tabbing}
	end \hrulefill\\\smallskip\\
}
\newenvironment{codetable}
{
	\smallskip\\\begin{tabular}{p{140pt}p{180pt}}
	\hline\\
}
{
	\\\hline
	\end{tabular}\smallskip\\
}
\newenvironment{codetablefig}[1]
{
	\begin{figure}[hbt]
	\caption{ #1 }
	\begin{tabular}{p{140pt}p{180pt}}
	\hline\\
}
{
	\\\hline
	\end{tabular}
	\end{figure}
	\pagebreak[4]
}
\newcommand{\codeexercise}[1]{\smallskip\\\large{Exercise.}\normalsize\quad #1\smallskip}
\chapter{Assembly with NAGT}
This is a little tutorial/HOWTO for writing Argante programs in NAGT assembly.
It is assumed you are already familiar with general programming, though not completely
familiar with Argante.\\
It's not completely essential to understand these low-level workings; there are higher
level languages available which hide these details. But if you do put in the brain sweat,
you will be able to break the limits of these languages, and even help develop them!
\section{Assembler Basics}
It's assumed you have your copy of Argante2 built and ready to run, other documents
should be better able to explain how to do this - and we are assuming some basic skills
here. We will also work out of the Argante root directory.
\subsection{Registers and Arithmetics}
Our first program will add the numbers from 1 to 5.
\begin{codefile}{ex\_reg1}
\# This program adds the numbers from 1 to 5.\\
\# This is a comment, by the way.\\
.code\\
\>	mov u:r0, 0\\
\>	add u:r0, 1\\
\>	add u:r0, 2\\
\>	add u:r0, 3\\
\>	add u:r0, 4\\
\>	add u:r0, 5\\
\>	syscall2 \$IO\_PUTINT, u:r0\\
\>	syscall2 \$IO\_PUTCHAR, 10\\
\>	halt
\end{codefile}
Save this code into a file called ex\_reg1.agt, using a text-only editor (very important!)  and \emph{assemble} it with "compiler/nagt ex\_reg1.agt ex\_reg1.img". This tells NAGT to produce ex\_reg1.img from ex\_reg1.agt, and in future this will be what we mean when we say 'assemble x.agt'.\\
Now run it - it should say 15.\\
So, what does each line do in the above code?\\
\begin{codetable}
\texttt{.code} & tells the assembler to expect code instructions.\\
\texttt{mov u:r0, 0} & sets the \emph{unsigned} value of register 0 to $0$.\\
\texttt{add u:r0, $x$} & adds $x$ to the unsigned value of register 0.\\
\texttt{syscall2 \$IO\_PUTINT, r0} & prints the contents of register 0, assuming
it to be an integer.\\
\texttt{syscall2 \$IO\_PUTCHAR, 10} & prints a newline (ASCII 10) character.\\
\texttt{halt} & terminates the program.\\
\end{codetable}\medskip\\
In an ordinary programming language, you construct variables, which have specific types.\\
In Argante2, there are 31 registers (32, but r31 is special) - basically variables -
which can be accessed as any type. 31 might not seem all that many, but you will
never end up using more than about 10 in even the most complicated programs, and
there are ways to create more variables, which we will go into later.\\
So why aren't there specific types to registers? As they are shared between all functions in your program, it would otherwise be impossible to use 32 ints in one place and 32 floats in another. Another good reason is that it dramatically simplifies the syscall calling convention - more on that later.\\
As an aside, an \emph{unsigned} value is one which can take integral (whole number)
values from $0$ to $4294967295$ (no negatives!), a \emph{signed} value can range from
$-2147483647$ to $+2147483647$ (integral), and a floating point value can take
positive or negative, fractional values from, at worst, $1*10^{-37}$ to $1*10^{37}$.
A float is not as fast as a integer type, and may only provide 6 digits of accuracy.
\codeexercise{Write a program to find 3 factorial and 5 factorial.}
\begin{codetablefig}{Summary of arithmetic commands.}
\multicolumn{2}{l}{The following operations work with all register types: u: (unsigned),}\\
\multicolumn{2}{l}{s: (signed) and f: (floating point), and any mixture.}\\
\texttt{mov $x$, $y$}&Sets $x$ to $y$\\
\texttt{add $x$, $y$}&Adds $y$ to $x$\\
\texttt{sub $x$, $y$}&Subtracts $y$ from $x$\\
\texttt{mul $x$, $y$}&Multiplies $x$ by $y$\\
\texttt{div $x$, $y$}&Divides $x$ by $y$\\
\texttt{mod $x$, $y$}&Sets $x$ to the remainder of $x / y$\\
\multicolumn{2}{l}{The following binary operations require two unsigned arguments}\\
\texttt{and $x$, $y$}&Sets $x$ to the binary intersect of $x$ and $y$\\
\texttt{or $x$, $y$}&Sets $x$ to the binary union of $x$ and $y$\\
\texttt{xor $x$, $y$}&Sets $x$ to the exclusive-or of $x$ and $y$\\
\texttt{not $x$}&Sets $x$ to its binary inverse.\\
\texttt{shl $x$, $y$}&Multiplies $x$ by $2^y$, discarding bits above $2^{32}$.\\
\texttt{shr $x$, $y$}&Divides $x$ by $2^y$, discarding bits below $1$.\\
\texttt{rol $x$, $y$}&Multiplies $x$ by $2^y$, moving bits above $2^{32}$ to the lower end.\\
\texttt{ror $x$, $y$}&Divides $x$ by $2^y$, moving bits below $1$ to the upper end.\\
\multicolumn{2}{l}{The following commands are useful for debugging.}\\
\texttt{syscall2 \$IO\_PUTINT, $x$}&prints $x$ as a signed int.\\
\texttt{syscall2 \$IO\_PUTCHAR, 10}&prints a newline character.\\
\end{codetablefig}
\subsection{Labels and Control Flow}
In the previous section, you can see we used the same code, repeated five times,
with different arguments. When you are dealing with complicated operations (generally
anything above five instructions) you don't want to do this: it's too much work,
it's easy to make mistakes, and one change has to be repeated throughout the code.\\
Hence, we use subroutines.\\
The following program uses 'call' and 'ret' to print the first ten terms of the Fibonacci Sequence:
\begin{codefile}{ex\_lab1}
\# This program prints ten terms of the Fibonacci Sequence.\\
.code\\
\>	mov u:r0, 0\\
\>	mov u:r1, 1\\
\>	syscall2 \$IO\_PUTINT, r1\\
\>	call :fibo\_func\\
\>	call :fibo\_func\\
\>	call :fibo\_func\\
\>	call :fibo\_func\\
\>	call :fibo\_func\\
\>	call :fibo\_func\\
\>	call :fibo\_func\\
\>	call :fibo\_func\\
\>	call :fibo\_func\\
\>	syscall2 \$IO\_PUTCHAR, 10\\
\>	halt\\
\>fibo\_func:\\
\>	mov u:r2, u:r0\\
\>	add u:r2, u:r1\\
\>	mov u:r0, u:r1\\
\>	mov u:r1, u:r2\\
\>	syscall2 \$IO\_PUTCHAR, 20 \# ASCII 20 - a space character.\\
\>	syscall2 \$IO\_PUTINT, u:r1\\
\>	ret 1
\end{codefile}
Type in this program, assemble and run it. What order are the instructions executed in?\\
\codeexercise{Write a program to find and print the second, fourth, eighth and sixteenth powers of 1, 2, 3 and 4.}\\
At this point we will mention the 'jmp' instruction. It resembles the 'call' instruction in syntax and function, but there is no 'ret' equivalent. The next 'ret' instruction will still return to the last unreturned 'call', which can be useful if the last instruction in a function would have been a call to another function:
\begin{codefile}{ex\_lab2}
\# This program demonstrates tricky (ab)use of the jmp instruction.\\
.code\\
\>	mov u:r0, 0\\
\>	mov u:r1, 1\\
\>	mov u:r2, 1\\
\>	call :func\\
\>	halt\\
:func\\
\>	call :subfunc\\
\>	call :subfunc\\
\>	call :subfunc\\
\>	call :subfunc\\
\>	jmp :subfunc\\
:subfunc\\
\>	mov u:r3, u:r1\\
\>	mul u:r3, u:r3\\
\>	add u:r0, u:r3\\
\>	add u:r1, u:r2\\
\>	syscall2 \$IO\_PUTINT, u:r0\\
\>	syscall2 \$IO\_PUTCHAR, 10\\
\>	ret 1
\end{codefile}
This example is a little contrived. What do you think would happen without the 'jmp' - try it!\\
\codeexercise{Modify the program above to add the squares of the first five even numbers.}
\begin{codetablefig}{Summary of flow control commands.}
\texttt{:label}&Defines a label\\
\texttt{jmp :label}&Makes :label the next instruction to execute.\\
\texttt{call :label}&Executes the instructions following :label, but resume here on a 'ret'.\\
\texttt{ret 1}&Resumes execution following the last active 'call'\\
\texttt{ret $x$}&Resumes execution following the $x$th last active 'call'. 'calls' which occurred after that will be forgotten.\\
\end{codetablefig}
\subsection{Conditionals and Loops}
So far you haven't done anything a pocket calculator wouldn't be able to do faster. We'll make up for that by introducing the conditional commands.
\begin{codefile}{ex\_cond1}
\# This program prints the sum of the numbers between 1 and 100.\\
.code\\
\>	mov u:r0, 0\\
\>	mov u:r1, 1\\
:loop\_top\\
\>	add u:r0, u:r1\\
\>	add u:r1, 1\\
\>	ifbel u:r1, 100\\
\>	jmp :loop\_top\\
\>	syscall2 \$IO\_PUTINT, u:r0\\
\>	syscall2 \$IO\_PUTCHAR, 10\\
\>	halt
\end{codefile}
That made life easy! Only when a conditional comparison is TRUE does the next statement
get executed. Note that execution normally continues on to the statement after that, so
unless the first statement is JMP, the second statement is always executed.\\
\codeexercise{What happens when you put one conditional right after another? Can you see a use for this?}\\
As loops like this are fairly common, there is a special 'loop' instruction, which
decrements a counter and jumps to an address if the counter is above zero.
\begin{codefile}{ex\_cond2}
\# This program prints the numbers between 1 and 100 ending in 4 or 6.\\
.code\\
\>	mov u:r0, 1\\
\>	mov s:r1, 100 \# Signed values for decreasing loops are preferable.\\
:loop\_top\\
\>	add u:r0, 1\\
\>	mov u:r2, u:r0\\
\>	mod u:r2, 10 \# What's the last digit? \\
\>	ifneq u:r2, 4\\
\>	ifeq u:r2, 6\\
\>	call :print\_num\\
\>	loop s:r1, :loop\_top\\
\>	halt\\
:print\_num\\
\>	syscall2 \$IO\_PUTINT, u:r0\\
\>	syscall2 \$IO\_PUTCHAR, 10\\
\>	ret 1
\end{codefile}
\codeexercise{Modify the above program to print an exclamation mark (ASCII 33) if the number is also divisible by 7.}
\begin{codetablefig}{Summary of conditionals.}
\texttt{ifabo $x$, $y$}&If $x <= y$, skip next command.\\
\texttt{ifbel $x$, $y$}&If $x >= y$, skip next command.\\
\texttt{ifneq $x$, $y$}&If $x = y$, skip next command.\\
\texttt{ifeq $x$, $y$}&MIf $x \neq y$, skip next command.\\
\texttt{loop $x$, :addr}&Decrease $x$, and goto :label if above zero.\\
\end{codetablefig}
\section{Advanced Assembler}
Hey, well done! Now you can do all the maths as you could ever want. Ok, maybe
not, but somebody is probably writing a complex/trig/log/linalg module somewhere.
Give us enough time and someone will write a Differential Equation engine...
\smallskip\\
Not that you wanted to do maths anyway, right?
\subsection{Data and References}
In the previous section, we mentioned that there were other ways of storing data
other than registers. All of them revolve around memory, so it's time to learn
to use it.\\
Here's one of our old programs - recognize it?
\begin{codefile}{ex\_dat1}
\# This program prints ten terms of the Fibonacci Sequence.\\
.data\\
:terma\\
\>	0\\
:termb\\
\>	1\\
:tempterm\\
\>	0\\
:count\\
\>	10\\
.code\\
:loop\_top\\
\>	syscall2 \$IO\_PUTINT, *s::termb\\
\>	call :fibo\_func\\
\>	loop *s::count, :loop\_top\\
\>	syscall2 \$IO\_PUTCHAR, 10\\
\>	halt\\
\>fibo\_func:\\
\>	mov *u::tempterm, *u::terma\\
\>	add *u::tempterm, *u::termb\\
\>	mov *u::terma, *u::termb\\
\>	mov *u::termb, *u::tempterm\\
\>	syscall2 \$IO\_PUTCHAR, 20\\
\>	ret 1
\end{codefile}
The first thing you should notice is that the first thing in the file is not '.code' - it's '.data'.
This must mean that all those labels are \emph{data} labels.\\
And, scattered through the code where the registers used to be, are their references - looking sort of like code labels, only with a '*' in front of them, and a 'u:' part we recognize as specifying the data type.\\
When we used the ':label' syntax for code, the assembler replaced the reference with the address of the code. Now for data, we don't really want to know the address of the variable - we want to change the contents. So, we must indicate that we want the contents of the address. That's the '*'.\\
You might think that using data labels instead of registers makes code a lot more readable, and you'd be right. But registers are accessed with less complexity than other memory, which makes them a lot faster.\\
You can also apply the dereference to registers:
\begin{codefile}{ex\_dat2}
\# This program prints an array.\\
.data\\
:numbers\\
\>	-1000\\
\>	2000\\
\>	-4000\\
\>	8000\\
\>	-16000\\
.code\\
\>	mov u:r0, :numbers\\
\>\# \% means "size (in integers) of this label's content"\\
\>	mov u:r1, \%numbers
:loop\_top\\
\>	syscall2 \$IO\_PUTINT, *s:r0\\
\>	syscall2 \$IO\_PUTCHAR, 10\\
\>	add u:r0, 1\\
\>	loop u:r1, :loop\_top\\
\>	halt
\end{codefile}
Note r0 is \emph{not} used as more than one type. The dereference line means
"get the signed contents of the memory at address r0", not "get the contents
of the memory at signed address r0". There's no such thing as a signed address.
\codeexercise{What happens if you change \%numbers in the above code to, say, 400? Why?}
\subsection{Strings and Buffers}
So far all we've done is print numbers, numbers, numbers. Maybe last section
we used an array - very important stuff, but still \emph{dull}. Nobody wants
code that runs like a collection of Windows error messages
("MMSYSTEM 451: Contact vendor for details").\\
So, onward! Text!\\
\begin{codefile}{ex\_str1}
\# Ultracool stuff.\\
.rodata\\
:message\_to\_world\\
\>	"You smell DISGUSTING!!$\backslash$n"\\
.code\\
\>	mov u:r0, :message\_to\_world\\
\>\# \bsz{} means "size (in bytes) of this label's content"\\
\>	mov u:r1, \bsz{}message\_to\_world\\
\>	syscall \$IO\_PUTSTRING\\
\>	halt
\end{codefile}
Well, it's original.\\
Note we have to feed IO\_PUTSTRING two arguments via registers - the
address and the size (in bytes), and that it isn't called by 'syscall2' -
it's called by 'syscall'. ('syscall2' is called what it is because it has two arguments.)\\
The second thing to notice is that we didn't use '.data' - we used
'.rodata'. Because this string is not supposed to be modified, we can
tell Argante so, and impede any evil hackers changing our manifesto.\\
\codeexercise{Write a program that prints an array of strings.}\\
\begin{codefile}{ex\_str2}
\# Ultracooler stuff.\\
.ropack\\
:question\_of\_user\\
\>	" What's your name?$\backslash$n"\\
:message\_to\_user\\
\>	"  You smell disgusting, "\\
.data\\
:buffer\\
\>	0x0 repeat 8\\
.code\\
\>	mov u:r1, :question\_of\_user\\
\>	mov u:r2, \bsz{}question\_of\_user\\
\>	syscall \$CFD\_WRITE\\
\>	mov u:r1, :buffer\\
\>	mov u:r2, \bsz{}buffer\\
\>	syscall \$CFD\_READ\\
\>	mov u:r3, \bsz{}buffer\\
\>	sub u:r3, u:r2\\
\>	mov u:r1, :message\_to\_user\\
\>	mov u:r2, \bsz{}message\_to\_user\\
\>	syscall \$CFD\_WRITE\\
\>	mov u:r1, :buffer\\
\>	mov u:r2, u:r3\\
\>	syscall \$CFD\_WRITE\\
\>	mov u:r1, 10\\
\>	syscall \$CFD\_WRITE\_CHAR\\
\>	halt
\end{codefile}
This program makes use of input, which is a relatively unpolished feature
as of the time of writing. (Windows is currently OK, unices depend on the
configure options. No Multithreading = Console Input works.) So it might
fail with a ERR\_BAD\_FD code.\\
\codeexercise{Get out your Syscall Reference and figure out how this thing works, when it works.}\\
As we have more than one string, and they are readonly, it's foolish to keep them
in separate sections in the file. So we specify '.ropack', which makes the assembler
pack them into the same section.
\begin{codetablefig}{Summary of data manipulators.}
\texttt{*}&References the contents of an address.\\
\texttt{:}&The address of a label\\
\texttt{\%}&The size of a label's content in dwords (four-byte blocks)\\
\texttt{\bsz{}}&The size of a label's content in bytes.\\
\texttt{.data}& Creates data segments.\\
\texttt{.rodata}& Creates readonly data segments.\\
\texttt{.ropack}& Place multiple items in one readonly data segment.\\
\texttt{.packed}& If you use this without knowing what you are doing, a black hole will swallow the earth.\\
\texttt{repeat $n$}& Repeat all data on this line $n$ times.\\
\end{codetablefig}
\subsection{Dynamic Allocation}
Now you are able to read input from a console, or, with a bit of side reading,
from a file, you're going to be able to make things happen. But what if you
don't know much memory your buffer is going to need? You could just put a really
big number after repeat, but this will bloat your .img file. What you need
is to be able to create data sections at runtime!\\
The trouble is that the assembler has no way of knowing where this segment is
going to end up. So you have to keep this address in a register, or, if you
store it in a data segment, you have to do two separate dereferences to
get at the contents.
\begin{codefile}{ex\_alc1}
\# What DOES this program DO?\\
.data\\
:numbers\\
\>	-1000\\
\>	2000\\
\>	-4000\\
\>	8000\\
\>	-16000\\
.code\\
\>	\# We'll store the address in r16.\\
\>	mov u:r16, \%numbers\\
\>	alloc u:r16, 3 \# 3 = 1 + 2 = READABLE + WRITABLE\\
\>	mov u:r1, u:r16\\
\>	mov u:r0, :numbers\\
\>	mov u:r2, \%numbers
:loop\_top\\
\>	mov *u:r1, *u:r0\\
\>	add u:r0, 1\\
\>	add u:r1, 1\\
\>	loop u:r2, :loop\_top\\
\>	free u:r16\\
\>	halt
\end{codefile}
\codeexercise{Create a small, useful example of alloc, and send it to the author.}
\begin{codetablefig}{Summary of data allocation commands.}
\texttt{alloc $x$, $y$}&Allocates a block of size $x$ and permissions $y$, and stores its address in $x$.\\
\texttt{realloc $x$, u:$y$}&Resizes block $x$ to size $y$ and stores its new address in $x$.\\
\texttt{realloc $x$, s:$y$}&Changes the permissions of block $x$ to $y$.\\
\texttt{free $x$}&Deallocates a block.\\
\end{codetablefig}
\subsection{The Metastack}
Let's say you have a subroutine which uses some temporary variables - in registers, or data areas,
it doesn't matter. And this subroutine has to call itself, multiple times, without messing up
its temporaries. Tricky? You should already know one solution using an array\ldots\\
Save the temporaries in the array before you call, and increment the address into it.
Then the next call will save its own temporaries in a different part of the array.
And when it returns, you decrement the address and restore your temporaries.\\
It's a lot of work.\\
Hence Argante provides built-in support for this with the metastack functions. Once
you have set up the metastack, you just 'push' (add) and 'pop' (retrieve and remove
from stack) things you want to save or retrieve,
\begin{codefile}{ex\_stk1}
\# Yes, they are getting boring for me, too.\\
.data\\
:stack\\
\>	0x0 repeat 40\\
.code\\
\>	stack :stack, \%stack\\
\>	mov u:r0, 1\\
\>	call :funny\_func\\
\>	syscall \$IO\_PUTINT\\
\>	halt\\
:funny\_func\\
\>	add u:r0, 1\\
\>	ifabo u:r0, 5\\
\>	ret 1\\
\>	push u:r0\\
\>	call :funny\_func\\
\>	pop u:r0\\
\>	ret 1\\
\end{codefile}
If everything works, the program will print $2$.
\begin{codetablefig}{Summary of metastack.}
\texttt{stack $x$, u:$y$}&Set the metastack address to $x$ and size to $y$.
Does not zero the metastack pointer - but this may change. \\
\texttt{push $x$}& Add $x$ on the stack and increment the metastack pointer.\\
\texttt{pop $x$}&Decrement the metastack pointer, and retrieve a value into $x$.\\
\end{codetablefig}
\subsection{Some Things We Skipped}
There is another sort of assembler directive we haven't mentioned.
These are placed before any '.data' or '.code' directives and set various
elements of the image header. And most of these elements are obsolete.\\
There is one command which still may have a future: '!signature'. Currently
it allows you to embed a short (32char) description of your program, which
should probably include its author.\\
If you have a register dedicated to a specific function all the
way through your code, you don't actually have to write the number
all the way through your code. Use '.define' macros, and your code will be
easier to read, and you will make less mistakes.\\
It's also a good idea to use '.define' macros for numeric options and
nearly anything that's shared between different bits of your program.
\begin{codefile}{ex\_msc1}
\# UselessProgram version 2.\\
!signature "UselessProgram \email{noflames@please}"\\
\# We'll still store the address in r16, but it's easy to change.\\
.define addy\% u:r16\\
.data\\
:numbers\\
\>	-1000\\
\>	2000\\
\>	-4000\\
\>	8000\\
\>	-16000\\
.code\\
\>	mov addy\%, \%numbers\\
\>	alloc addy\%, 3 \# 3 = 1 + 2 = READABLE + WRITABLE\\
\>	mov u:r1, addy\%\\
\>	mov u:r0, :numbers\\
\>	mov u:r2, \%numbers
:loop\_top\\
\>	mov *u:r1, *u:r0\\
\>	add u:r0, 1\\
\>	add u:r1, 1\\
\>	loop u:r2, :loop\_top\\
\>	free addy\%\\
\>	halt
\end{codefile}
\codeexercise{Why use a funny symbol on the end of addy? What happens if
you change 'addy\%' to 'rs'? Why?}
\section{Argante Concepts}
This ends the NAGT-specific part of this guide. From here on, this applies to
everything in Argante, on every level - even HLLs.
\subsection{Exceptions}
When a 'fatal' error happens in Argante, we use a concept usually found
in object oriented languages: that of \emph{raising} (or throwing) an exception.
If there is no active \emph{exception handler} in a subroutine in which an
exception is raised, the subroutine is aborted and the exception propagates up
to the parent routine. (If there is no parent routine, the program dies.)\\
Examples speak louder than words, anyway:
\begin{codefile}{ex\_exc1}
\# Exception demonstrator.\\
.ropack\\
:ex1\\
\>	"Exception handler 1 activated.$\backslash$n"\\
:ex2\\
\>	"Exception handler 2 activated.$\backslash$n"\\
.code\\
\>	handler :hand2\\
\>	call :subrtn1\\
\# The code should never get here, so let's crash if it does.\\
\# Handler 0 deactivates the exception handler.\\
\>	handler 0\\
\>	mov *u::ex1, 0 \# write to readonly data\\
:subrtn1\\
\# This handler is for the current function only.\\
\>	handler :hand1\\
\# To throw an exception yourself, use 'raise'.\\
\>	raise 0xfeedface\\
\>	ret 400 \# This should die, too, despite a handler...\\
:hand1\\
\>	mov u:r0, :ex1\\
\>	mov u:r1, \bsz{}ex1\\
\>	syscall \$IO\_PUTSTRING\\
\# The exception handler has been activated,\\
\# and so won't trigger again (that would loop forever!)\\
\>	raise 0xf007f00d\\
\>	ret 400\\
:hand2\\
\# The exception number (0xf007f00d) will be placed in r31.\\
\>	syscall2 \$IO\_PUTHEX, u:r31\\
\>	syscall2 \$IO\_PUTCHAR, 10\\
\>	mov u:r0, :ex2\\
\>	mov u:r1, \bsz{}ex2\\
\>	syscall \$IO\_PUTSTRING\\
\>	halt
\end{codefile}
Exception numbers for system-defined errors are listed in include/exception.h,
and their symbolic names (\$ERR\_xxxx) can be used in assembler code.
\begin{codetablefig}{Summary of exception commands.}
\texttt{handler 0}&Any exceptions in this routine will be passed upward. (No handler.)\\
\texttt{handler :label}&Any exceptions in this routine will set r31 to the
exception number, and jump to :label.\\
\texttt{raise $x$}&Raises exception $x$.\\
\end{codetablefig}
\subsection{Syscalls}
Each syscall is part of an Argante kernel module (i.e. privileged code) which are
written to allow interactions with the real system or to do things difficult to
achieve from Argante code. (For example, the StrFD module could be written in
Argante, but it would be slower and would require an OO language to be as useful.)\\
In general, all syscalls (used with the 'syscall' op) will take their first argument
in r0, their second argument in r1, their third argument in r2, and so on.\\
For a call marked by "SYS2", you can use the second argument of syscall2 instead of
r0.\\
Return values are not so consistent. When you see 'ignore' in a list of outputs from
a syscall, it means that register is returned unchanged.\\
Take CFD\_READ for example:\\
it returns\\
$[$ ignore, unsigned @null\_after\_data, unsigned space\_left\_in\_buffer $]$
\\
which means r0 is unchanged.\smallskip\\
All the syscalls can be found in the Argante2 Syscall Reference - likely to
be found wherever you found this document.
\section{Conclusion}
Hey, you now know every opcode in Argante2 assembler. Go forth, and code!
\end{document}
