\documentclass[a4paper,oneside,openany]{book}
\title{\Huge{Argante2}\\\huge{Design Notes}}
\author{
	\LARGE{James Kehl}\\
	ecks@optusnet.com.au
}
\begin{document}
\maketitle{}
\newcommand{\email}[1]{$<$#1$>$}
\newcommand{\FIXME}{\texttt{**FIXME**}}
\newcommand{\perjsk}{James Kehl \email{ecks@optusnet.com.au}}
\newcommand{\permz}{Michal Zalewski \email{lcamtuf@coredump.cx}}
\newcommand{\bsz}{\^ }
\newcommand{\rarrow}{$\rightarrow$}

\chapter{Kernel Design}
\section{"Ring 0" - core essentials}
\subsection{Memory Access}
\begin{verbatim}
typedef struct {
  union {
    unsigned long u;
    signed long s;
    a2float f;
  } val;
} anyval;
\end{verbatim}
a2float is a floating-point type with sizeof(a2float) == sizeof(long) - float on
32bit systems, double on 64bit systems.\smallskip\\
Within Argante, \emph{all} memory is of this type, and accesses are always aligned
to anyval block boundaries; this makes it easier to catch boundary violations, but
makes string manipulation difficult and leads to 16 GB of addressable memory on
32bit systems (which have an 4 GB address space).\smallskip\\
The central part of Argante is securing memory access - preventing code
modification, call stack overwriting, heap corruption attacks, and so on
(ad nauseum, sadly).\smallskip\\
Currently the memory space is divided into 16384 blocks of 1 MB each. This has the
sole advantage of simplicity. It's likely to change, as 1 MB is not a Goldilocks
number - it's either way too small or way too large, but any changes will require
changing the image file format.
\newpage
\begin{verbatim}
#define A2_MEM_READ                1
#define A2_MEM_WRITE               2
#define A2_MEM_MAPPED              4

struct memblk {
  unsigned int mode;		/* Access permissions */
  anyval* memory;		/* Real pointer */
  unsigned int size;		/* Size */
  int destroy_scnum;		/* Syscallnum to do a basic FREE */
  unsigned alib_id;		/* so we can get rid of lib's mem */
};
\end{verbatim}
Memory pages have the type shown above, though they should not be accessed
directly. MAPPED memory is accessible like normal memory, but cannot be resized
or have its permissions changed. Freeing it will call the syscall given in
'destroy\_scnum' if it is nonzero.\smallskip\\
size is in units of anyval.\smallskip\\
alib\_id is provided to allow the dynamic linker to free pages that were loaded
from an image file which was subsequently removed. Pages allocated at runtime will
not be recognized.
\begin{verbatim}
unsigned mem_alloc(struct vcpu *curr_cpu, unsigned size,
        unsigned flags);
unsigned mem_realloc(struct vcpu *curr_cpu, unsigned addr,
        unsigned newsize);
void mem_changeperm(struct vcpu *curr_cpu, unsigned addr,
        unsigned newflags);
void mem_dealloc(struct vcpu *curr_cpu, unsigned addr);
\end{verbatim}
The above functions will allocate, resize, free or change the permissions of a block
of memory, or throw an exception if they fail. When they return an unsigned value,
it is an address from Argante-space, i.e. usable in mem\_ro or mem\_rw...
\begin{verbatim}
const anyval *mem_ro(struct vcpu *curr_cpu, unsigned addr);
anyval *mem_rw(struct vcpu *curr_cpu, unsigned addr);
\end{verbatim}
The above functions permit access to a single anyval for readonly or read-write usage.
This is faster than paired get\_mem\_value/set\_mem\_value calls, but kernel code must
not override the 'const' and write to memory that was checked out as readonly. Use \_rw,
this checks for write permission.
\begin{verbatim}
#define dwords_of_bytes(a) \
        ((a / sizeof(anyval)) + ((a % sizeof(anyval)) ? 1 : 0))

const anyval *mem_ro_block(struct vcpu *curr_cpu, unsigned addr,
        unsigned dwords);
anyval *mem_rw_block(struct vcpu *curr_cpu, unsigned addr,
        unsigned dwords);
\end{verbatim}
The above functions the permissions and validity of a large block of memory at once. It is
possible to cast the return value to a char * or const char * if needed.\smallskip\\
dwords\_of\_bytes gives the number of anyval units required for 'a' bytes, and is useful
for calculating the 'dwords' arguments of the \_block functions.
\begin{verbatim}
int kerntoa_strcpy(struct vcpu *curr_cpu, unsigned addrto,
        int size, const char *from);
int kerntoa_memcpy(struct vcpu *curr_cpu, unsigned addrto,
        const char *from, int size);
int atokern_memcpy(struct vcpu *curr_cpu, char *to,
        unsigned addrfrom, int size);
\end{verbatim}
kerntoa\_strcpy copies a null-terminated string into Argante-space. The rest are
similar and should be self-explanatory.
\subsection{Opcode Design}
Argante has a fixed bytecode size. Without this, it is impossible to really know what
the code does. On other architectures, one perfectly ordinary-looking instruction can
become another if you jump into the middle of it, and that also alters every
instruction executed after that.
\begin{verbatim}
struct bcode_op {
  unsigned char bcode;
  unsigned char type;
  short reserved; /* 32-align arguments for speed */
  anyval a1;
  anyval a2;
};
\end{verbatim}
Consider an instruction like ADD. Logically, it should work for signed, unsigned
and floating-point types, and each combination needs a different code to implement.\smallskip\\
Also, you should be able to use a constant, a register or the contents of a memory
address, but that doesn't change the actual instruction, it only changes the location
of the arguments in real memory, and you still need to know the type of the argument.
\begin{verbatim}
#define TYPE_UNSIGNED	000
#define TYPE_SIGNED	001
#define TYPE_FLOAT	002

#define TYPE_IMMEDIATE	000
#define TYPE_REGISTER	004
#define TYPE_POINTER	010

#define TYPE_A1(a)	((a) << 0)
#define TYPE_A2(a)	((a) << 4)
#define TYPE_VALMASK	(TYPE_UNSIGNED | TYPE_SIGNED | TYPE_FLOAT)
\end{verbatim}
The first 2 bits of a bcode\_op's 'type' field are used for the type of argument 1,
the next 2 are flags designating immediate values, registers, pointers, or pointers
within registers. The next 4 are like the last 4, only for argument 2.\smallskip\\
The type bits are used to find what instruction to call. The flags specify how to
get to the arguments - in the case of immediates, it's just the address of the anyval
within the bcode\_op structure.\smallskip\\
To stop people overwriting immediates, the instructions which change an argument
need to be known about. When an image file is loaded, \\
validate\_bcode\_page from imageman.c is run. This checks that each referenced
instruction actually exists, that it doesn't try and overwrite immeduates, that
it doesn't use register 33, and finally precalculates the offsets within the 
JIT table that each instruction will require (hey, I could remove the addition
too! \FIXME)\smallskip\\
The precalculation's for speed, not security, but it does prevent anything that
changes an opcode from having any effect.\smallskip\\
There's a script that does all the hard work of generating the JIT table and
tracking which arguments of which instructions are readonly and read-write. It
gets its information from lines like:
\begin{verbatim}
/*! MOV 2 u RW u RO = cmd_mov_uu */
\end{verbatim}
The above lines merely says that \emph{all} MOV instructions have 2 arguments,
the first of which is written to and the second of which is readonly; and that
cmd\_mov\_uu is the name of the function which handles 2 unsigned arguments.\smallskip\\
It really should have an opcode number in there, too. \FIXME\smallskip\\
cmd\_mov\_uu, written fully, is just:
\begin{verbatim}
static void cmd_mov_uu (struct vcpu *curr_cpu, anyval *a1,
  anyval *a2) {
  a1->val.u=a2->val.u;
}
\end{verbatim}
\subsubsection{Syscall2 - the rationale}
As instructions are all of equal size, a one argument operation takes the same
space as a two argument operation - the one argument op is wasting 4 bytes.
\smallskip\\
In the case of syscalls, which can require any number of arguments, and take
them in r0 - r30, the problem gets worse - it's 4 bytes wasted in the 1-argument
syscall instruction itself, and another 12 on a MOV instruction to pass another
argument to the syscall.\smallskip\\
Equally, when debugging asm using IO\_PUTINT, you have to back up r0, move the
number to print into r0, perform the syscall, and restore r0. And hope you
didn't clobber something when backing up r0 in the first place. Annoying, ugly,
and a potential cause of errors.\smallskip\\
So, the new SYSCALL2 instruction takes two arguments, and passes its second
argument to the syscall function.\smallskip\\
This is nice when writing lots of asm, but high level languages will find it
easier to ignore, and it's not a very significant optimization when it comes
to large, multiple-argument syscalls. It's also argued that it makes code
analysis harder, though the assembler warning when a non-syscall2 function is
syscall2'd doesn't leave much confusion for me.\smallskip\\
Perhaps syscall2 will be "voted off the island". Stay tuned.
\subsubsection{Future Changes}
The following opcodes were badly thought out and should change in some future
version. This will probably happen at the same time as adding a 'address' field
to data in the image format (and numerous other fixes).
\begin{itemize}
\item{}alloc (takes u:size, u:perms, writes address to size) is badly thought out. Who
wants freshly-created read only memory? It should be more like realloc (addr, size).
\item{}realloc (takes u:addr, s:perms) should be renamed. This is the change-permissions
opcode and will never change its arguments, unlike realloc (u:addr, u:size).
\item{}setstack will also be improved. There is currently no way to set the pointer,
and resizing could be made cleaner. More thought required on this.
\end{itemize}
\subsection{Exceptions}
So many programmers are too lazy to check return values. And in most cases, they
don't really care, because if one call fails the rest of the function can't keep
going.\smallskip\\
Hence the rise of exceptions. But as it's hard enough checking return values,
running CHECK\_FAILURE\_FN when it's required is damn near impossible. So
throw\_except is implemented with longjmp, and you have to be able to deal
with not having your function finish properly. Don't acquire a lock or 
malloc() a block you might not be able to release. alloca() is better.\smallskip\\
Subexception handlers may be possible if this causes real problems.\smallskip\\
Also, exceptions can only be thrown from within a running VCPU. This means
functions which are used outside this context - say, image\_load, or the
vcpu\_start or vcpu\_stop functions of a module - can't throw exceptions.\\
(Note: we could allow exceptions in those functions. We'd just need a handler!)
\subsection{Multithreading}
When you are running VCPUs and interpreting keyboard input all from one
thread, it becomes painfully difficult to avoid blocking calls. Under Unix,
some functions are not available in nonblocking varieties (unless you
want to rewrite them, that is), like readline() or gethostbyname().\smallskip\\
So, you can't do it all from one thread.\smallskip\\
One burden this places on you is to make sure you don't cross threading lines.
Writing to static variables from VCPU code is out, as is poking at other 
images. Poking at running VCPUs from manager (main-thread) code is also out.
For most big modifications (module loads/unloads) the VCPUs have to be spun
down.\smallskip\\
The other burden is that to spin down a VCPU means anything more specific about
where a process is apart from curr\_cpu-$>$IP is lost. There should either be
only one cancellation point per syscall (so the operation is atomic: done or
not done) or a register should be modified to track the process. Otherwise we
will 'stutter' - very untraceably.
\section{"Ring 1" - syscalls, modules, and userspace API}
\subsection{Modules}
To keep the non-optional (and therefore most security-critical, for there is
no way to change the statically linked code barring a recompile) parts of
Argante as small (and understandable/reviewable) as possible, the system
interactions are partitioned up into modules. On most systems, these are
dynamically loadable; on the rest, they can still be changed with ./configure
options.\smallskip\\
From the stats, just as many people download the Windows, non-dl binaries as
the source, so static linking is needed as a fallback...\smallskip\\
Four functions are needed in every module:
\begin{itemize}
\item\texttt{int module\_init(unsigned lid)} is called when the module is loaded.
Executed in manager context, so is guaranteed to be the only running part of
the module: so setting globals is OK.\smallskip\\
lid is 'library id', a unique identifier for your library which acts as a key
into the reserved structure and FD tables. Store this in a static global if you
need it later.\smallskip\\
Return 0 for success and 1 for failure - if 1, module is unloaded and syscalls
stay unavailable.
\item\texttt{void module\_shutdown(void)} is called when the module is unloaded.
Again, this is the only part of module running when this is called.
\item\texttt{void module\_vcpu\_start(struct vcpu *vcpu)} is called when a new\\
VCPU starts. Other VCPUs may be using this module at this stage, and exceptions
are not allowed. (No code has been executed and so no handlers can be set.)
\item\texttt{void module\_vcpu\_stop(struct vcpu *vcpu)} is called when a VCPU stops.
Again, no exceptions allowed - this is already a dead CPU, and that could revive it.
Bad.
\end{itemize}
These hooks have proved sufficient, so far. Anything requiring more details
should probably be written into the core kernel.
\subsection{Reserved Structures}
Due to multithreading requirements, you cannot use static/global variables to keep
data about a particular VCPU. So, to store per-VCPU data, we have the 'reserved
structure' array, which provides one void pointer per VCPU per module. You can
store whatever you like in it; a malloc()d structure being most appropriate here.
\smallskip\\
The library ID passed to module\_init is needed to use these functions. Most of
the time you'll only use set\_reserved in vcpu\_start. Be sure to free in
vcpu\_stop!
\begin{verbatim}
void *module_get_reserved(struct vcpu *cpu, int lid);
int module_set_reserved(struct vcpu *cpu, int lid,
  void *newdata);
\end{verbatim}
set\_reserved returns 1 if it fails, so you can deallocate the data and assume
crash position. One possible cause of failure is a corrupt LID. Another is
that the system is out of memory.\smallskip\\
get\_reserved may return NULL if set\_reserved hasn't been called or it failed.
If it returns NULL, your best bet is to throw an exception (I recommend OOM).
The alternative to aborting is to dereference a NULL pointer and bring the
system down, so always check the return value!
\subsection{Syscalls}
\begin{verbatim}
#define SYSCALL_ARGS struct vcpu *curr_cpu, const anyval *arg
typedef void syscallfunc (SYSCALL_ARGS);

extern int register_syscall(unsigned id, syscallfunc *f);
extern int unregister_syscall(unsigned id);
\end{verbatim}
register\_syscall and unregister\_syscall should be called during module\_init
and module\_shutdown (and ONLY then). These create and destroy the association
between a numeric syscall ID and the function it should be dispatched to.
This is stored in a hash table.
\smallskip\\
The function takes the VCPU pointer (of course) and arg - which is a \emph{readonly}
argument - either r0 if the function's been syscalled, or the second argument
of a syscall2.
\subsection{Autogenerator}
As with the JIT table, syscall functions are so - well, homogenous - that it's
worth automating the syscall registration/deregistration and maintainance of the
compiler's ID\rarrow{}syscall tables.\smallskip\\
To do this you'll need to be writing in C (I don't know ADA, nor do I know anyone
else who'll admit to knowing it :P). Using comment-bang lines (like for the JIT)
a perl script will do the rest, complete with \#ifdefs so that static linking works.
\begin{verbatim}
static inline int module_internal_init(int lid):
static inline void module_internal_vcpu_start(struct vcpu *cpu):
static inline void module_internal_vcpu_stop(struct vcpu *cpu):
static inline void module_internal_shutdown(void);
\end{verbatim}
These should be static at the very least, if not static inline.
They are \emph{internal} and should not be available externally, and most
especially not to other modules which might be linked with yours.\smallskip\\
To include the autogenerated code which the kernel will be interfacing
with, write
\begin{verbatim}
#include "file_name.h"
\end{verbatim}
where file\_name is the name of your module. Any existing header of that name
will be overwritten. (I suppose the file name could have been \texttt{file\_name.hgen}
\ldots)
\begin{enumerate}
\item\texttt{/*! allowed x - y */}
\item\texttt{/*! NEW\_CALL1 = new\_call1 */}
\item\texttt{/*! NEW\_CALL1 99900 = new\_call1 */}
\item\texttt{/*! NEW\_CALL1 SYS2 = new\_call1 */}
\item\texttt{/*! NEW\_CALL1 99900 SYS2 = new\_call1 */}
\end{enumerate}
The first line tells the autogenerator that this module has been assigned a
range of syscall numbers. The range 99900 - 99999 is defined as a testing range:
feel free to use this before you've been assigned a range, but if you release
code that uses it, Sendmail ("the daemon from hell") will come round and have a
little chat with you.\smallskip\\
The second line tells the autogenerator that the syscall named NEW\_CALL1 is
implemented in new\_call1, and that the autogenerator should pick a number for
it. This number should be written into the code ASAP. Think what happens if
FS\_RENAME's number became that of FS\_DELETE's.\smallskip\\
The third line is the same as the second, only with the number specified.
This is good.\smallskip\\
The next two lines are the same as the previous two, only they say that
new\_call1 uses its arg parameter and so is a SYSCALL2.
\subsection{Heirarchical Access Control}
I hope everyone understands ACLs (access control lists) and inherited permissions.
(If not, you'd better ask the guy who's r00ting your box right now.)
\smallskip\\
To control access to resources, Argante uses a method called Heirarhical
Access Control. This defines what operations may be performed on what
resources (be they files, sockets, or system statistics). These resources
make up a filesystem-like tree. The operations, too, are defined to form
a tree: so we have /open/, /open/read/, /open/write/, and
/open/write/overwrite.
\smallskip\\
For files, the resource name is NOT the filename. There needs to be a
'namespace' prefix: /fs/etc/hosts is the file /etc/hosts, for example.
/tcp4/192.168.0.1 is an IPv4 host, very distinct from /fs/192.168.0.1,
which is a (yawn) file.
\smallskip\\
So that every permission does not have to be specified for every file,
inheritance is used. If at a particular level, a permission is unspecified,
it is inherited from a more general level. (The operations tree is searched
before the resource tree.)
\smallskip\\
Argante2's HAC is a little different from Argante1's. The HAC is actually
stored in a heirarchical structure (actually doubly nested hashed linklists,
though someday it will change to something less memory hungry) rather than
being implemented as string comparisons.
\smallskip\\
This means you can specify HAC rules in any order whatsoever, and you can't
specify wildcards: the generalization of rules has to be done during the
module design phase. Entries like /file/create and /file/delete should be
avoided in favour of /create/file and /create/directory - assuming you will
grant permissions for directory/file creation more ofen than file
creation/deletion\ldots
\smallskip\\
Remember - every 'directory' is a new layer, consuming memory, and taking
time to traverse. Don't be unnecessarily specific: do you really need all
the details in /fs/fops/local/create/file/regular? /fs/ should be in the
resource path, to start with.
\smallskip\\
Argante2's HAC DOES NOT check for wildcards and directory traversals. This
is not its job. The filenames '*' and '..lck' are valid filenames, and in
namespaces other than the filesystem '..' itself might be safe and '\#!'
dangerous. To cut out directory traversals, the function fold() is
provided. (Don't stick the namespace prefix on before this, or someone
can jump namespaces.)
\smallskip\\
\texttt{VALIDATE(dir, atype)} is the usual way to check HAC permissions;
this macro requires curr\_cpu to be in a variable called 'curr\_cpu', of
all things. If your syscalls stick to SYSCALL\_ARGS then you're set.
\smallskip\\
If VALIDATE fails, it throws an exception. So don't malloc() or open a file
or do anything that might not get cleaned up before you call it. alloca() is
preferred over malloc() for just this reason. Beware, though, of alloca()'ing
large chunks of memory; it fails disasterously (Chernobyl-style).
\smallskip\\
If you can't use VALIDATE, \texttt{validate\_access(curr\_cpu, dir, atype)}
is the underlying call. It returns nonzero on access failure.
\subsection{Virtual File Descriptors}
This is designed to be used for, like the name suggests, virtual file handles.
The VFD facilities have significant advantages over reserved structures for
this purpose, and have their own limitations which aren't significant for this
use.
\smallskip\\
One very significant motivation for using VFDs is that you don't need to mess
with lots of resizing of reserved structures. The limitation is that each VCPU
has an upper limit on virtual file descriptors, which are shared between all
modules. (Of course, this is an intentional, adjustable limit.)
\begin{verbatim}
int vfd_alloc_new(struct vcpu *curr_cpu, int lid);
\end{verbatim}
This returns a unique number which identifies your new VFD. Your OPEN call is
pretty useless if it doesn't return this to the user.
\begin{verbatim}
void *vfd_get_data(struct vcpu *curr_cpu, int lid, int handle);
void vfd_set_data(struct vcpu *curr_cpu, int lid, int handle,
  void *newd);
void vfd_dealloc(struct vcpu *curr_cpu, int lid, int handle);
\end{verbatim}
These all throw a ERR\_BAD\_FD exception if passed a handle that wasn't created
by your library (according to lid).
\begin{verbatim}
int vfd_find_mine(struct vcpu *curr_cpu, int lid);
\end{verbatim}
vfd\_find\_mine is worth noting; it returns a handle if your module owns a VFD.
It's most useful for destroying all your VFD's in a vcpu\_stop routine - mind,
though, that you actually vfd\_dealloc or vfd\_find\_mine will keep returning the
same number...
\smallskip\\
Despite the name, these aren't limited to IO. You could write a MySQL module and
use VFDs for database connections, or maybe even queries. You might write a
hashtable module and use VFDs for particular hashtables. Anything that a VCPU might
want multiples of is a candidate.
\smallskip\\
There is also a common layer for VFDs that DO stick to the I/O paradigm, that
allows CFD\_WRITE to write to files or consoles or sockets or syslog or\ldots
depending on what module created the VFD, and also allows modules to send/read
kernel-space data via VFDs - so PUT\_HEX does not get implemented 20 times...
\subsection{Common Operations layer}
\begin{verbatim}
typedef void cfdop_close_fd (struct vcpu *curr_cpu, void *vfd);

/* for agents to create VFDs. */
typedef int cfdop_create_fd (struct vcpu *curr_cpu,
  const char *desc, int in, int out);

/* returns "Block size" - maximum size to read/write at once. */
typedef int cfdop_start (struct vcpu *curr_cpu, void *vfd);

/* returns bytes read/written */
typedef int cfdop_write_block (struct vcpu *curr_cpu, void *vfd,
  const char *buf, int size);
typedef int cfdop_read_block (struct vcpu *curr_cpu, void *vfd,
  char *buf, int size);

/* CFD operations table version 1. */
struct cfdop_1 {
  cfdop_start       *read_start;
  cfdop_start       *write_start;
  cfdop_read_block  *read_block;
  cfdop_write_block *write_block;
  cfdop_close_fd    *fd_close;
  /* A unique endian-independant code (ie a string) for
     accepting agent VFD's. Only used for fd_create. */
  int fd_desc;
  cfdop_create_fd   *fd_create;
};
\end{verbatim}
To implement CFD, you put some function addresses into this table (If your
VFDs are always RO, then you might leave some fields NULL. Very few modules
implement agent-FDs, and most leave fd\_create NULL.
\smallskip\\
The way it works is: say CFD\_READ is called. It looks up what module that
VFD came from, and finds the table that's associated with that module. From
that table, read\_start is called with the VFD data. It might, say, check the
HAC. Then it returns the block size writes should occur in (mostly for
historical reasons). The block size MUST be a multiple of sizeof(anyval).
Then read\_block is called repeatedly until all the data is read.
\smallskip\\
The agent-fd calls are to allow management agents to create VFDs for consoles,
or GUI connections, through a module which can speak the protocol (in this case,
VT100 or X). The fd\_create call will soon change to accept flexsocks instead
of file descriptors for portability.
\smallskip\\
The following calls allow searching and registering tables. \_lid\_set should
be called during module\_init.
\begin{verbatim}
/* get/set a table by lid */
extern void cfdop1_lid_set(unsigned lid,
  const struct cfdop_1 *a);
extern const struct cfdop_1 *cfdop1_lid_get(unsigned lid);
/* get a table by a fd_desc - for agent-fds */
extern const struct cfdop_1 *cfdop1_fddesc_get(int fddesc);
/* Get the table for a vfd */
extern const struct cfdop_1 *cfdop1_vfd_get(struct vcpu 
  *curr_cpu, unsigned handle);
\end{verbatim}
\subsection{"Alib": dynamic linking}
This should be avoided in favour of IPC. However, there is no IPC
functionality yet.
\section{"Ring 2" - management}
\subsection{Agents}
TBA. It's safe to say anyone who's ever used the A2 console knows there needs
to be a better way...
\subsection{FlexSock}
For a module to talk via an agent-FD, it has to have some way of writing and
reading data with that FD. But is it a file handle, a FILE *, a Win32 named
pipe, or an OpenSSL connection, each of which have their own write method?
\smallskip\\
Hence - the FlexSock abstraction layer to isolate all the messy details.
OpenSSL has something similar in its BIO functionality - wouldn't it be nice
if everyone had OpenSSL!
\smallskip\\
Details, as usual, in flux.
\subsection{Remote IPC - DRAFT}
\subsubsection{Routing}
Nobody has written a single line of rIPC code for this version yet, so this
is just ideas... forgive the lack of ASCIIs.\smallskip\\
In \emph{any} network, you have two types of element - the hub and the node.
A node cannot connect directly to another node, nor can it connect to more
than one hub. If it connected to multiple hubs, it would be represented as a
node connected to a hub connected to the other hubs.\smallskip\\
In A2 rIPC, a node's only function is computation, and a hub's only function
is routing and communication. The only overlapping functionality is in the
common network protocols, the lower levels of which should be encapsulated
within FlexSock. Node\rarrow{}hub, hub\rarrow{}node and hub\rarrow{}hub protocols
will not have much in common.\smallskip\\
All comms are in network byte order (big-endian).\smallskip\\
When a new node connects to a hub, all other hubs need to know how to send
messages to it (the other nodes just depend on their hub for this). The new node's
hub sends a ADDED message to all the hubs it is connected to.
\begin{verbatim}
struct msg_ADDED {
  uint16_t node_address;
  uint16_t hops;
  uint16_t round_trip_time;
}
\end{verbatim}
Node addresses are 16 bits for forced obsolescence. If the address space is ever
exhausted within a single rIPC network (!), it will be a lot less problematic to
extend at that stage than an exhausted 32 bit address space. As a hub will never
be an endpoint for a message, they are not rIPC-addressable.\smallskip\\
When a hub recieves an ADDED message, it adds one to 'hops', increases the 'rtt'
by the rtt of the sender, and files the new data and sender under the node address.
If the sender already advertised this node, the previous data is replaced.\smallskip\\
A hub MUST not accept a direct connection using a node address which the hub
already has a path for (especially a direct, hops=0 one, though the hub SHOULD
check if the original connection died). The only time this will happen is when
two nodes use the same address, which is an accident or an attack.\smallskip\\
If the new data is 'better' in terms of rtt than the last 'best' for that address,
the new data is sent out to all hubs (except the sender). Hubs MAY delay (quench)
this retransmission if the data is changing quickly.\smallskip\\
Hubs MAY ignore ADDEDs that exceed their limits on hops or rtt. If so, these limits
should be chosen carefully to avoid "one-way streets". Hubs MAY also forget the
'worst' records if they have too many, although they will never have more than one
per hub they are connected to, and MUST retain at least two records. Hubs MAY allow
a 'fuzz factor' in order to recycle common data.\smallskip\\
When a hub connects to an existing network, it MUST broadcast ADDEDs for all the
nodes already in its table. This is to allow existing networks to be joined.
\begin{verbatim}
struct msg_DROPPED {
  uint16_t node_address;
}
\end{verbatim}
When a node disconnects from its hub, the hub broadcasts a DROPPED message.
When a connection between hubs is lost, every path using the lost hub must be
removed. Any nodes which the hub no longer has a path for then generate a
DROPPED message, which prompts other hubs to forget that path, and rebroadcast
the DROPPED message if they have no path to that hub. Quenching MUST NOT occur
here.\smallskip\\
So, when a hub gets a DROPPED message, it removes the entry for the source
from the node's paths, and if the hub has no more paths for that node, it tells
all the other hubs that it cannot find that node with more DROPPED messages.\smallskip\\
If a node's best path has been lost, but the hub has a path remaining, it MUST
send all hubs it knows (even the one which sent the DROPPED), an ADDED message
containing the new data. Otherwise the fastest, or even only, route might be
overlooked.
\begin{verbatim}
struct msg_DATA {
  uint16_t dst_address;
  uint16_t src_address;
  uint16_t hops;
  char data[];
}
\end{verbatim}
When a hub recieves a data packet, it accepts a sacred responsibility to get
that packet to someone else. First, it finds the server with the lowest listed
rtt for dst\_address, and attempts to hand the message off. If that fails, the
hub\rarrow{}hub connection is considered dead, and DROPPEDs generated. And then
the hub tries the next fastest route.\smallskip\\
If the hub ends up running out of routes, the DATA becomes a DATA\_REJECT, src
and dst addresses are switched, hops are set to zero, and the data is dropped.
If a DATA\_REJECT is rejected, then a warning should be generated and the
packet MUST be dropped.
\subsubsection{Routing - Caveats}
It is impossible for a rogue node to fiddle the routes, and a node attempting a
DoS by adding and removing itself can be catered for by quenching - introducing
a time lag into ADDED rebroadcasting.\smallskip\\
However, a rogue hub could easily steal data and cause DoS. The only solutions
are configuring each hub to only accept certain other hubs, or to use
authentication. Either solution relies on trust. Hub\rarrow{}hub connections
should use SSL in any case. Node\rarrow{}hub connections are likely to be within
computers and secure networks, so SSL is mostly overkill. (But flexibility never
hurts).
\end{document}
