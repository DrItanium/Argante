\documentclass[a4paper,oneside,openany]{book}
\title{\Huge{Argante2}\\\huge{Syscall Reference}}
\author{
	\LARGE{James Kehl}\\
	ecks@optusnet.com.au
}
\begin{document}
\maketitle{}
\newcommand{\email}[1]{$<$#1$>$}
\newcommand{\FIXME}{\texttt{**FIXME**}}
\newcommand{\perjsk}{James Kehl \email{ecks@optusnet.com.au}}
\newcommand{\permz}{Michal Zalewski \email{lcamtuf@coredump.cx}}
\newcommand{\bsz}{\^ }
\newcommand{\rarrow}{$\rightarrow$}

\newcommand{\scargs}[1]{[ #1 ]}
\newenvironment{scmodule}[4]
{
	\pagebreak[4]
	\section{#1}
	\begin{tabbing}
	\enskip\=\qquad\qquad\qquad\qquad\=\+\kill
	\emph{Author:}\>#2\nopagebreak\\
	\emph{Editor:}\>#3\nopagebreak\\
	\emph{Last Updated:}\>\nopagebreak#4
	\end{tabbing}
	\smallskip
}
{
\smallskip\\
}
% Whooee.
\newenvironment{sccall}[2][(SYS2)]
{
	\subsection[#2]{#2 #1}
}
{
%\\\hrulefill
}
\chapter{Syscall Reference}
Most syscalls will take their first argument in r0, second argument in r1, and so on. A syscall marked SYS2 will use an extra syscall2 argument as its first input, and take any extra arguments from r1 on.\\
The syntax used to describe inputs and outputs is borrowed from LAC. A "string" is
two arguments: a buffer address, and that buffer's size.
\begin{scmodule}{ALIB}{\perjsk}{\perjsk}{Sun Jan 13 2002}
Provides syscalls for run-time linking.
\end{scmodule}
Known Issues:\\
ALIB\_LOOKUP doesn't say whether symbol is code or data.
\begin{sccall}[]{ALIB\_OPEN}
  Inputs: \scargs{"library-filename"}\\
 Outputs: \scargs{unsigned library\_handle}\\
  SideFx: Loads code, data \& symbols from specified image.\\
   Other: Can throw ERR\_ALIB\_FAIL
\end{sccall}
\begin{sccall}{ALIB\_LOOKUP}
  Inputs: \scargs{unsigned library\_handle, "symbol name"}\\
          (A zero library-handle means any)\\
 Outputs: \scargs{unsigned symbol\_address}\\
  SideFx: N/A\\
   Other: Throws ERR\_ALIB\_FAIL on a bad lib handle\\
	  Throws ERR\_ALIB\_NOSYM if symbol not found or undefined
\end{sccall}
\begin{sccall}{ALIB\_CLOSE}
  Inputs: \scargs{unsigned library\_handle}\\
 Outputs: \scargs{}\\
  SideFx: Unloads pages, data \& symbols belonging to specified lib\\
   Other: Throws ERR\_ALIB\_FAIL on a bad lib handle
\end{sccall}

\begin{scmodule}{CFD}{\perjsk}{\perjsk}{Sat 25 May 2002}
The Common File Descriptor operations module allows generic access to virtual
file descriptors.
\end{scmodule}
Known Issues:\\
Not all calls are implemented yet.
\begin{sccall}{CFD\_CLOSE}
Inputs: \scargs{unsigned vfd\_handle}\\
Outputs: \scargs{}\\
SideFx: Closes and invalidates vfd\_handle.\\
Other: Throws ERR\_BAD\_FD if file descriptor does not support this operation.
\end{sccall}
\begin{sccall}{CFD\_READ}
Inputs: \scargs{unsigned vfd\_handle, "writable buffer..."}\\
Outputs: \scargs{ignore, unsigned @null\_after\_data, unsigned space\_left\_in\_buffer}\\
SideFx: Reads up to r2 bytes from vfd\_handle\\
Other: Throws ERR\_BAD\_FD if file descriptor does not support this operation.
\end{sccall}
\begin{sccall}{CFD\_WRITE}
Inputs: \scargs{unsigned vfd\_handle, "buffer..."}\\
Outputs: \scargs{ignore, unsigned @null\_after\_data, unsigned bytes\_unwritten}\\
SideFx: Writes up to r2 bytes from vfd\_handle\\
Other: Throws ERR\_BAD\_FD if file descriptor does not support this operation.
\end{sccall}
\begin{sccall}{CFD\_WRITE\_NT}
Inputs: \scargs{unsigned vfd\_handle, "bufferbufferbuffer$\backslash$000"}\\
Outputs: \scargs{ignore, unsigned @null\_after\_data, unsigned bytes\_unwritten}\\\
SideFx: Writes r2 bytes or up to the first null, whatever comes first.\\
Other: Throws ERR\_BAD\_FD if file descriptor does not support this operation.
\end{sccall}
\begin{sccall}{CFD\_WRITE\_CHAR}
Inputs: \scargs{unsigned vfd\_handle, unsigned char}\\
Outputs: \scargs{}\\
SideFx: Writes low byte of char to VFD.\\
Other: Throws ERR\_BAD\_FD if file descriptor does not support this operation.
\end{sccall}
\begin{sccall}{CFD\_WRITE\_FLOAT}
Inputs: \scargs{unsigned vfd\_handle, float a, unsigned min\_digits,\\unsigned max\_digits}\\
Outputs: \scargs{}\\
SideFx: Writes a to VFD in "\%g" format.\\
Other: Throws ERR\_BAD\_FD if file descriptor does not support this operation.\\
Throws ERR\_ARG\_TOOLONG if string would be more than 32 characters, or max\_digits or min\_digits are larger than 8.
\end{sccall}
\begin{sccall}{CFD\_WRITE\_INT}
Inputs: \scargs{unsigned vfd\_handle, signed a}\\
Outputs: \scargs{}\\
SideFx: Writes a to VFD as a signed int.\\
Other: Throws ERR\_BAD\_FD if file descriptor does not support this operation.\\
Throws ERR\_ARG\_TOOLONG if string would be more than 16 digits.
\end{sccall}
\begin{sccall}{CFD\_WRITE\_UINT}
Inputs: \scargs{unsigned vfd\_handle, unsigned a}\\
Outputs: \scargs{}\\
SideFx: Writes a to VFD as a unsigned int.\\
Other: Throws ERR\_BAD\_FD if file descriptor does not support this operation.\\
Throws ERR\_ARG\_TOOLONG if string would be more than 16 digits.
\end{sccall}
\begin{sccall}{CFD\_WRITE\_HEX}
Inputs: \scargs{unsigned vfd\_handle, unsigned a}\\
Outputs: \scargs{}\\
SideFx: Writes a to VFD in hexadecimal format.\\
Other: Throws ERR\_BAD\_FD if file descriptor does not support this operation.\\
Throws ERR\_ARG\_TOOLONG if string would be more than 16 digits.
\end{sccall}
\begin{sccall}{CFD\_WRITE\_OCT}
Inputs: \scargs{unsigned vfd\_handle, unsigned a}\\
Outputs: \scargs{}\\
SideFx: Writes a to VFD in octal format.\\
Other: Throws ERR\_BAD\_FD if file descriptor does not support this operation.\\
Throws ERR\_ARG\_TOOLONG if string would be more than 16 digits.
\end{sccall}

\begin{scmodule}{DISPLAY}{Unknown}{\perjsk}{Fri Jul 13 2001}
Simple console output module for debugging.
\end{scmodule}
Known Issues:\\
This will be phased out when agents are introduced.\\
Does not flush output until newline printed.
\begin{sccall}[]{IO\_PUTSTRING}
  Inputs: \scargs{"string"}\\
 Outputs: \scargs{}\\
  SideFx: Displays string on kernel console.\\
   Other: N/A
\end{sccall}
\begin{sccall}{IO\_PUTINT}
  Inputs: \scargs{signed a}\\
 Outputs: \scargs{}\\
  SideFx: Displays signed integer on kernel console.\\
   Other: N/A
\end{sccall}
\begin{sccall}{IO\_PUTCHAR}
  Inputs: \scargs{unsigned a}\\
 Outputs: \scargs{}\\
  SideFx: Displays raw 8 low bits of a on console.\\
   Other: "syscall2 \$IO\_PUTCHAR, 10" will print newlines and flush output. 
\end{sccall}
\begin{sccall}{IO\_PUTFLOAT}
  Inputs: \scargs{float a}\\
 Outputs: \scargs{}\\
  SideFx: Displays float in "\%g" format.\\
   Other: N/A 
\end{sccall}
\begin{sccall}{IO\_PUTHEX}
  Inputs: \scargs{unsigned a}\\
 Outputs: \scargs{}\\
  SideFx: Displays hex number.\\
   Other: N/A 
\end{sccall}

\begin{scmodule}{FS}{\perjsk}{\perjsk}{Fri Jul 13 2001}
Interaction with filesystem and IO to real files.
\end{scmodule}
ERR\_ARG\_TOOLONG is thrown when a filename is larger than PATH\_MAX.\\
ERR\_BAD\_FD occurs when you try and read or write something you didn't
open for that (incl. directories).\\
Known Issues:\\
Does not support nonblocking IO.\\
Open flags are possibly more confusing than real system's.\\
Truncation doesn't happen yet. \FIXME\\
Partial write/read only returns errors, doesn't throw them.\\
ERR\_GENERIC happens a little too often.\FIXME
\begin{sccall}{FS\_OPEN}
  Inputs: \scargs{unsigned open\_mode, "filename"}\\
 Outputs: \scargs{unsigned vfd\_handle}\\
  SideFx: Creates file if not existing.\\
   Other: Access modes:\\
   	  0 - APPEND mode\\
	  1 - READ flag\\
	  2 - WRITE flag\\
	  4 - FSEEK flag\\
	  Opening a file for plain WRITE will truncate the file. (soon.)
\end{sccall}
\begin{sccall}{FS\_OPEN\_EXISTING}
  Inputs: \scargs{unsigned access\_mode, "filename"}\\
 Outputs: \scargs{unsigned vfd\_handle}\\
  SideFx: Fails with ERR\_FILE\_NOT\_EXIST if not existant.\\
   Other: Access modes as above.
\end{sccall}
\begin{sccall}{FS\_OPEN\_CREATE}
  Inputs: \scargs{unsigned access\_mode, "filename"}\\
 Outputs: \scargs{unsigned vfd\_handle}\\
  SideFx: Creates file, fails with ERR\_FILE\_EXIST if file already exists.\\
   Other: Access modes as above.
\end{sccall}
\begin{sccall}{FS\_READ}
  Inputs: \scargs{unsigned vfd\_handle, "writable buffer..."}\\
 Outputs: \scargs{ignore, unsigned @null\_after\_data, unsigned space\_left\_in\_buffer}\\
  SideFx: Reads at most r2 bytes into buffer.\\
   Other: N/A.
\end{sccall}
\begin{sccall}{FS\_WRITE}
  Inputs: \scargs{unsigned vfd\_handle, "buffer to write..."}\\
 Outputs: \scargs{ignore, unsigned @null\_after\_data, unsigned bytes\_unwritten}\\
  SideFx: Writes r2 bytes to given file.\\
   Other: N/A.
\end{sccall}
\begin{sccall}{FS\_FLUSH}
  Inputs: \scargs{unsigned vfd\_handle}\\
 Outputs: \scargs{}\\
  SideFx: All bytes written are immediately stored, and any bytes written by another process are read (if applicable).\\
   Other: N/A.
\end{sccall}
\begin{sccall}{FS\_SEEK}
  Inputs: \scargs{unsigned vfd\_handle, unsigned new\_offset}\\
 Outputs: \scargs{}\\
  SideFx: Current file position (for reading/writing) or directory handle will be set to new\_offset. Sparse files may be created.\\
   Other: Only does SEEK\_SET.
\end{sccall}
\begin{sccall}{FS\_TELL}
  Inputs: \scargs{unsigned vfd\_handle}\\
 Outputs: \scargs{unsigned file\_pos}\\
  SideFx: None.\\
   Other: May be used on directories, but value is not guaranteed to be consistent.
\end{sccall}
\begin{sccall}[]{FS\_WD\_GET}
  Inputs: \scargs{"writable buffer for working directory"}\\
 Outputs: \scargs{ignore, unsigned path\_length}\\
  SideFx: Puts working directory path in buffer.\\
   Other: N/A.
\end{sccall}
\begin{sccall}[]{FS\_WD\_SET}
  Inputs: \scargs{"new working directory"}\\
 Outputs: \scargs{}\\
  SideFx: Changes working directory\\
   Other: N/A.
\end{sccall}
\begin{sccall}[]{FS\_OPEN\_DIR}
  Inputs: \scargs{ignore, "directory name"}\\
 Outputs: \scargs{unsigned vfd\_handle}\\
  SideFx: None past the obvious.\\
   Other: N/A.
\end{sccall}
\begin{sccall}{FS\_READ\_DIR}
  Inputs: \scargs{unsigned vfd\_handle, "writable buffer..."}\\
 Outputs: \scargs{ignore, ignore, unsigned filename\_length}\\
  SideFx: Puts the next filename in buf.\\
   Other: N/A.
\end{sccall}
\begin{sccall}[]{FS\_MAKE\_DIR}
  Inputs: \scargs{"filename"}\\
 Outputs: \scargs{}\\
  SideFx: You reckon?\\
   Other: N/A.
\end{sccall}
\begin{sccall}[]{FS\_STAT}
  Inputs: \scargs{"filename"}\\
 Outputs: \scargs{unsigned file\_type, unsigned file\_size, unsigned mod\_time}\\
  SideFx: None.\\
   Other: Filetype is -\\
           0 - error\\
	   1 - regular file\\
	   2 - directory\\
	   3+- something else. (pipe?)
\end{sccall}
\begin{sccall}[]{FS\_RENAME}
  Inputs: \scargs{"source filename", "destination filename"}\\
 Outputs: \scargs{}\\
  SideFx: One hopes so.\\
   Other: Unlike C/Unix function of same name, fails if destination already exists.
\end{sccall}
\begin{sccall}[]{FS\_DELETE}
  Inputs: \scargs{"filename"}\\
 Outputs: None.\\
  SideFx: File/(empty) directory liquidation.\\
   Other: None.
\end{sccall}

\begin{scmodule}{LOCALLIB}{\permz}{\perjsk}{Fri Jul 13 2001}
Returns useful information on local system, and a few related utility functions.
\end{scmodule}
Known Issues:\\
ERR\_GENERIC happens a bit too often. \FIXME\\
RS\_STAT is Linux only and somewhat dusty anyway.\\
VS\_STAT is Argante1 only.
\begin{sccall}[]{LOCAL\_GETTIME}
  Inputs: \scargs{}\\
 Outputs: \scargs{unsigned seconds\_since\_epoch, unsigned $\mu{}$secs}\\
  SideFx: None.\\
   Other: N/A.
\end{sccall}
\begin{sccall}{LOCAL\_TIMETOSTR}
  Inputs: \scargs{unsigned seconds\_since\_epoch, "writable buffer"}\\
 Outputs: \scargs{ignore, ignore, unsigned bytes\_stored}\\
  SideFx: 'ctime' into buffer.\\
   Other: N/A.
\end{sccall}
\begin{sccall}[]{LOCAL\_GETHOSTNAME}
  Inputs: \scargs{"writable buffer"}\\
 Outputs: \scargs{ignore, unsigned bytes\_stored}\\
  SideFx: None.\\
   Other: Any hostname longer than 64 characters is 'bogus' in more ways than one.
\end{sccall}
\begin{sccall}[]{LOCAL\_GETRANDOM}
  Inputs: \scargs{}\\
 Outputs: \scargs{unsigned random\_integer}\\
  SideFx: None.\\
   Other: N/A.
\end{sccall}
\begin{sccall}[]{LOCAL\_RS\_STAT}
  Inputs: \scargs{}\\
 Outputs:\\
\scargs{unsigned uptime, unsigned 5sec\_load\_average, unsigned total\_ram\_in\_K,
 unsigned free\_ram, unsigned total\_swap, unsigned free\_swap, unsigned processes}\\
  SideFx: None.\\
   Other: N/A.
\end{sccall}

\begin{scmodule}{STRFD}{\perjsk}{\perjsk}{Sat 5 May 2002}
The StrFD module allows manipulation of a block of memory using a read / write / seek
paradigm and the Common File Descriptor operations set, and is probably the easiest
way to access memory on a byte-by-byte basis.
\end{scmodule}
Known Issues:\\
ERR\_STRFD\_BOUNDS occurs when the offset exceeds the allocated size.\\
ERR\_STRFD\_SEARCHFAIL occurs when one of the search functions fail.
The offset is also invalid when this occurs and should be reset.
\begin{sccall}[]{STRFD\_OPEN}
Inputs: \scargs{"existing buffer"}\\
Outputs: \scargs{unsigned vfd\_handle}\\
SideFx: Use this block as a static, unresizable StrFD.\\
Other: N/A.
\end{sccall}
\begin{sccall}[]{STRFD\_CREATE}
Inputs: \scargs{unsigned initial\_size}\\
Outputs: \scargs{unsigned vfd\_handle}\\
SideFx: Create a new memory block attached to a dynamically-grown StrFD.\\
Other: N/A.
\end{sccall}
\begin{sccall}{STRFD\_CLOSE}
Inputs: \scargs{unsigned vfd\_handle}\\
Outputs: \scargs{}\\
SideFx: Closes the given StrFD.\\
Other: Does NOT free dynamically-allocated memory, so be sure to get the address first.
\end{sccall}
\begin{sccall}{STRFD\_GET\_OFFSET}
Inputs: \scargs{unsigned vfd\_handle}\\
Outputs: \scargs{unsigned offset}\\
SideFx: None.\\
Other: This returns the byte offset within the memory block for which the next
read or write will start from.
\end{sccall}
\begin{sccall}{STRFD\_SET\_OFFSET}
Inputs: \scargs{unsigned vfd\_handle, unsigned offset}\\
Outputs: \scargs{}\\
SideFx: None.\\
Other: This sets the byte offset for which the next read or write will start from.
\end{sccall}
\begin{sccall}{STRFD\_GET\_ADDR}
Inputs: \scargs{unsigned vfd\_handle}\\
Outputs: \scargs{unsigned addr}\\
SideFx: None.\\
Other: This is most useful for accessing dynamically allocated strings outside
the StrFD module.
\end{sccall}
\begin{sccall}{STRFD\_GET\_SIZE}
Inputs: \scargs{unsigned vfd\_handle}\\
Outputs: \scargs{unsigned size}\\
SideFx: None.\\
Other: This is the flip side of STRFD\_GET\_ADDR.
\end{sccall}
\begin{sccall}{STRFD\_READ}
Inputs: \scargs{unsigned strfd\_handle, "writable buffer..."}\\
Outputs: \scargs{ignore, unsigned @null\_after\_data, unsigned space\_left\_in\_buffer}\\
SideFx: Reads up to r2 bytes from strfd\_handle and shifts the offset.\\
Other: N/A.
\end{sccall}
\begin{sccall}{STRFD\_WRITE}
Inputs: \scargs{unsigned strfd\_handle, "buffer..."}\\
Outputs: \scargs{ignore, unsigned @null\_after\_data, unsigned bytes\_unwritten}\\
SideFx: Writes up to r2 bytes from strfd\_handle and shifts the offset\\
Other: N/A.
\end{sccall}
\begin{sccall}{STRFD\_GETCHAR}
Inputs: \scargs{unsigned strfd\_handle}\\
Outputs: \scargs{unsigned a\_byte}\\
SideFx: Increments the offset.\\
Other: N/A.
\end{sccall}
\begin{sccall}{STRFD\_SETCHAR}
Inputs: \scargs{unsigned strfd\_handle, unsigned a\_byte}\\
Outputs: \scargs{}\\
SideFx: Writes the single byte and increments the offset.\\
Other: N/A.
\end{sccall}
\begin{sccall}{STRFD\_STRCHR}
Inputs: \scargs{unsigned strfd\_handle, signed shift\_val, unsigned needle}\\
Outputs: \scargs{}\\
SideFx: Shifts the offset until STRFD\_GETCHAR would return needle.\\
Other: Throws ERR\_STRFD\_SEARCHFAIL if character not found.
\end{sccall}
\begin{sccall}{STRFD\_STRSTR}
Inputs: \scargs{unsigned strfd\_handle, signed shift\_val, "needle"}\\
Outputs: \scargs{}\\
SideFx: Shifts the offset until STRFD\_READ would return needle.\\
Other: Throws ERR\_STRFD\_SEARCHFAIL if string not found.
\end{sccall}
\begin{sccall}{STRFD\_STRCMP}
Inputs: \scargs{unsigned strfd\_handle, "needle"}\\
Outputs: \scargs{signed difference}\\
SideFx: None - does not change offset.\\
Other: Returns any differences between needle and the next r2 bytes of the StrFD, so
could also be called strncmp or memcmp.
\end{sccall}
\begin{sccall}{STRFD\_SPN}
Inputs: \scargs{unsigned strfd\_handle, "accept"}\\
Outputs: \scargs{}\\
SideFx: Increases offset until STRFD\_GETCHAR would return a character not in "accept".\\
Other: Not Implemented. \FIXME
\end{sccall}
\begin{sccall}{STRFD\_CSPN}
Inputs: \scargs{unsigned strfd\_handle, "reject"}\\
Outputs: \scargs{}\\
SideFx: Increases offset until STRFD\_GETCHAR would return a character in "reject".\\
Other: Not Implemented. \FIXME
\end{sccall}

\end{document}
